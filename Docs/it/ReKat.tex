\documentclass{article}
% librerie per lingua e fonts
\usepackage[italian]{babel}
\usepackage[utf8]{inputenc}
\usepackage[T1]{fontenc}

%sitture matematica
\usepackage{amsmath}
\usepackage{amsfonts}
\usepackage{amssymb}
\usepackage{amsthm}
\usepackage{yhmath}

% informazioni sul compito
\title{ReKat Ita}
\author{Eldron 64}
\date{November 2023}

\begin{document}
	
\maketitle
Con questo documento cercherò di mettere in ordine tutto quello che il Kat Projekt ora ReKat ha rappresentato per me.

\section{Introduzione}
Nell'implementazione corrente il Kat è un insieme di Api che servono alla realizzazione di un progetto che spiegherò nel corso di questo rapporto.

Per il momento il Kat è divisibile nei seguenti engine:

\begin{itemize}
	\item Online: Node Network
	\item Synth: Sound System
	\item Graphik: Graphics Api
\end{itemize}

\section {Online}
Questa Api è divisa in due implementazioni:

\begin{itemize}
	\item Online Peer to Peer
	\item Online Server Side
\end{itemize}

Entrambe i Sistemi utilizziamo il metodo di criptazione RSA e si scambiano le chiavi pubbliche una volta connessi o al server o ad un nodo.

\subsection{Peer to Peer}
Questa implementazione si basa su un sistema di comunicazione Peer to Peer. Dove i vari nodi sono indipendenti tra loro e ognuno possiede tutte le funzionalità degli altri nodi. Ma può fornire servizi che gli altri nodi non supportano.

L'utilizzo di questo modulo è molto semplice e lineare:

All'inizio del programma viene creato un Nodo locale a cui chiunque si può connettere. Successivamente ci si può connettere ad un altro nodo nella rete.

I vari nodi possono eseguire dei programmi differenti ed offrire servizi diversi. Nel caso un servizio non sia disponibile in un nodo a cui si è collegati si può eseguire una ricerca all'interno del Network per trovare il nodo con il programma richiesto ed accedervi tramite i nodi intermediari.

\subsection{Server}
In questa implementazione il fulcro del Network è il server che esegue il programma principale a cui i vari client si connettono.

Questa implementazione rende più funzionale l'aggiornamento della rete, visto che esiste un unico provider di servizi. Questo approccio limita la libertà dei singoli nodi.

Inoltre non è consentito il collegamento diretto tra i client, ogni comunicazione tra di loro passa per il Server. Questa implementazione è necessaria al fine di permettere la comunicazione tra client anche se il client ricevente non è connesso, infatti la comunicazione viene salvata nel Server per poi essere inoltrata al client.

\subsection{sicurezza RSA}
Ogni comunicazione uscente da un client, nodo o server viene criptata tramite il sistema RSA, viene inoltre criptata ad ogni passaggio. Quando il messaggio deve passare per un intermediario viene criptato due volte una per il destinatario e una per l'intermediario. In modo tale da fornire all'intermediario le informazioni necessarie a passare il messaggio.

Il pacchetto che deve andare da A a B passando per C. verrà sottoposto ai seguenti passaggi:

\begin{itemize}
	\item A -> Pacchetto non criptato
	\item A -> Pacchetto criptato per B
	\item A -> Aggiunta al Pacchetto criptato le informazioni di spedizione per B
	\item A -> Nuovo Pacchetto criptato per C
	\item A-C -> A manda il pacchetto a C
	\item C -> Decriptazione del pacchetto ricevuto
	\item C -> Lettura dal pacchetto l'indirizzo di B
	\item C -> Rimozione delle informazioni di B
	\item C-B -> C manda il pacchetto a B
	\item B -> Decriptazione del pacchetto.
	\item B -> Pacchetto non criptato
\end{itemize}

Il sistema RSA si basa su un sistema a chiave pubblica e privata, le chiavi sono formate da una parte comune N e una parte propria E (per la pubblica) e D (per la privata).
Questi valori sono utilizzati per eseguire operazioni matematiche ai dati che devono essere compresi tra 1 e N, se non rispettano questa condizione verranno corrotti, per prevenire questa evenienza i dati vengono ridimensionati in sotto pacchetti di dimensione inferiore a N criptati e poi riuniti, questo processo funzione in entrambe le direzioni: criptazione <=> decriptazione.

A livello formale:

Dati P e Q primi.
Ricavo $N = P \cdot Q$
Il toziente $\phi = ( P - 1 ) ( Q - 1 )$ serve per trovare la chiave di criptazione.

Per trovare la chiave di criptazione E devo trovare quel numero tale che $ 1 < E < \phi $ e $ MCD ( E, \phi ) = 1 $

Per trovare D devo calcolare l’inverso moltiplicativo di E in modulo $\phi$. Per risolvere questo problema utilizzare il metodo di Euclide.
Il metodo di Euclide serve a trovare quel numero D tale che $D * E = 0 ( mod \phi )$.

Trovati questi tre numeri possiamo procedere alla criptazione: 

\begin{itemize}
	\item Dato il messaggio M
	\item lo converto in un numero $M_n$ ( utilizzando la notazione ascii posso associare ad ogni lettera un numero che può essere al massimo 255, detto questo posso riscrivere il messaggio come un numero in base 256 )
	\item il numero così costruito viene elevato alla potenza E modulo N $ M_c = M_n^E ( mod N )$ 
	\item $M_c$ sarà il nostro messaggio criptato
\end{itemize}

Per ricavare M da $M_c$ devo:

\begin{itemize}
	\item $M_d = M_n^D ( mod N )$
	\item successivamente posso utilizzare la conversione da numero a carattere utilizzando la tabella ascii e ottengo il messaggio M.
\end{itemize}
 
\section {Synth}
Questa libreria fornisce un modello di generazione sonora basato sul linguaggio umano e sui suoni che un essere umano può fare. Oltre ad una semplice interfaccia per riprodurre suoni.

Ci sono due approcci a questo problema.

Il primo consiste nella registrazione diretta di qualcuno che emette i suoni e alla loro successiva lavorazione e modulazione. Con questo approccio possiamo estrarre le componenti basilari della dizione da una serie di campioni per poi adattati all'utilizzatore.

Il secondo metodo consiste nella simulazione di un apparato vocale, questo approccio è quello più complesso in termini di realizzazione ma è estendibile ad altri ambienti sonori. Questo metodo non verrà trattato in questo documento.

\subsection{concetti di base}
Per iniziare dobbiamo definire un suono:

Un suono è una variazione di pressione che viene percepita da un apparato predisposto a misurare questa variazione. Un suono esiste in funzione dell’ascoltatore.

Un suono ha 4 componenti:
\begin{itemize}
	\item Il Tono
	\item La forma
	\item La durata
	\item La frequenza
\end{itemize}

Per semplificazione consideriamo il suono come quantizzato. Ovvero verrà rappresentato come una serie di valori che simbolizzano la pressione dell’aria in un determinato momento.

\subsubsection{il tono}
Il tono è il componente fondamentale di un suono. Ed è la più piccola parte periodica da cui è composto un suono.

Quando parliamo di vocalizzazione nell'essere umano possiamo dire che ogni fonema pronunciabile possiede un suo tono caratteristico che viene ripetuto per la durata della sua riproduzione e che ha come frequenza l’intonazione del suono stesso.

Date queste precisazioni diamo al tono l'onere di dare ad ogni fonema il suo nome. Il fonema è influenzato dalla forma della bocca, dalla posizione della lingua, dalla nasalità del suono etc.

In termini matematici possiamo indicare $T(t)$ come la pressione dell'aria in quel momento, dobbiamo inoltre dire che $T(t)$ è periodica e per convenzione ha come intervallo di ripetizione $2\pi$. Può essere inoltre scomposta in una serie di $sin(t)$ e $cos(t)$ secondo la trasformata di Fourier. $T(t)$ è una funzione normalizzata ovvero: $-1 \le T(t) \le 1$.

\subsubsection{La forma}
La forma di un fonema è il modo in cui la sua intensità varia nel tempo ovvero come inizia il suono, come viene mantenuto, e come decresce.

Queste caratteristiche definiscono la pronuncia di un suono e un suo accento caratteristico.

Un suono può crescere molto velocemente, può avere un effetto vibrato durante il mantenimento e può scendere molto lentamente.

La funzione della forma è composta secondo la seguente legge:

$$F(t) :
\begin{cases}
	C(t)\hspace{0.2cm}per\hspace{0.2cm}t \le 1 \\
	M(t)\hspace{0.2cm}per\hspace{0.2cm}1 < t \le 2\\
	D(t)\hspace{0.2cm}per\hspace{0.2cm}2 < t
\end{cases}
$$

Dove C,M,D sono le funzioni che indicano come $F(t)$ cresce nel tempo. Per convenzione e comodità di calcolo impongo $0 \le t \le 3$ per valori di X esterni al Dominio $F(t)$ restituisce un valore nullo. il valore di $t = 0$ indica l'inizio del suono mentre $t = 3$ ne indica la fine. Inoltre $\int_0^3F(t)dt = 1$.

\subsubsection{La durata}
Con la durata in tendiamo tre momenti: il tempo di crescita, il tempo di mantenimento e il tempo di decrescita.

Se con la forma descriviamo il modo in cui cresce e si modifica con la durata diamo informazioni su quanto tempo impiega per fare questa variazione.

La funzione Durata: $H(t)$ indica la velocità con cui cresce la funzione. come $F(t)$ è definita a pezzi:
$$H(t) :
\begin{cases}
	\frac{t}{t_1} \hspace{0.2cm}per\hspace{0.2cm} t \le t_1 \\
	\frac{t}{t_2} \hspace{0.2cm}per\hspace{0.2cm} t_1 < t \le t_2\\
	\frac{t}{t_3} \hspace{0.2cm}per\hspace{0.2cm} t_2 < t
\end{cases}
$$
dove $t_1$, $t_2$ e $t_3$ sono i tempi caratteristici delle durate.

detto questo possiamo riscrivere $F(t)$

$$F_H(t) :
\begin{cases}
	C(\frac{t}{t_1}) \hspace{0.2cm}per\hspace{0.2cm} t \le t_1 \\
	M(\frac{t}{t_2}) \hspace{0.2cm}per\hspace{0.2cm} t_1 < t \le t_2\\
	D(\frac{t}{t_3}) \hspace{0.2cm}per\hspace{0.2cm} t_2 < t
\end{cases}
$$

con $0 \le t \le t_3$ e $t_1 < t_2 < t_3$

\subsubsection{La frequenza}
La frequenza descrive l'altezza o intonazione di un suono ovvero quanto è acuto o grave suddetto suono.

Per completezza devo tenere in conto che l'uomo può variare la frequenza di un fonema mentre lo esegue quindi:

Definiamo $f(t)$ la frequenza di un suono nell'intervallo $t$ questa semplice funzione crea un'enormità di problemi relativi alla sincronizzazione dei toni con frequenza diversa.
In pratica quello che fa questa funzione è modificare il periodo della funzione $T(t)$, modificando il periodo possiamo ottenere una curva non continua questo perché prese due frequenze vicine $f_1$ e $f_2$ non possiamo dire che nel momento del cambio di frequenza $sin(f_1t) = sin (f_2t)$ questo crea il problema della sincronizzazione.

Per semplificare impongo $f(t) = K$, ma verrà sistemato in un iterazione successiva.

Ora dopo aver definito tutte le funzioni posso scrivere la funzione del Fonema come:

$$\Gamma(t) = F_H(t) T(2\pi f(t) t)$$

con $0 \le t \le t_3$.

\subsection{pronuncia}
Per fare in modo che una macchina pronunci correttamente una parola dobbiamo capire come noi pronunciamo le parole.

Noi come esseri umani possiamo emettere suoni singoli sotto la forma di fonemi o unire più fonemi ottenendo delle parole e con l'accostamento di più parole otteniamo una frase.

In pratica per pronunciare una parola bisogna pronunciare i vari fonemi che la compongono unendoli un qualche modo. Ma prima bisogna dividere le parole in fonemi.

\subsubsection{divisione fonetica}
Questo è un tema molto complesso. Infatti ogni lingua ha un suo sistema di dizione interno. Per comodità utilizzeremo l’International Phonetic Alphabet essendo il metodo di notazione fonetica più utilizzato e considerato più valido e completo.

Per quanto riguarda la divisione in fonemi bisogna implementare un sistema specifico per ogni lingua. Di seguito riporterò alcuni esempi:

\begin{itemize}
	\item Italiano \\
    	L'italiano ha una fonetizzazione abbastanza semplice, l'unica difficoltà risiede nelle lettere $g$ e $c$ che assumono suoni diversi in base al contesto:
    	\begin{itemize}
        	\item $c$ e $g$ dolci: si hanno quando la lettera è seguita da $i$ o $e$; esempi: $nici$, $bici$, $ciao$, etc.
        	\item $c$ e $g$ dure: si hanno quando la lettera non è seguita direttamente da $i$ o $e$; esempi: $gola$, $lingua$, $ghiro$, etc.
    	\end{itemize}
	\item Giapponese \\
    	Il giapponese possiede una serie di ideogrammi che rappresentano tutte le possibili sillabe e con queste costruiscono tutte le parole, questi caratteri vengono fonetizzati sempre allo stesso modo, ne consegue che basta costruire queste singole sillabe e poi unirle.
	\item Inglese \\
    	L'inglese a differenza delle precedenti lingue presenta una fonetizzazione molto particolare, per fonetizzare l'inglese esiste l' IPA (international Phonetic Alphabet), basandosi su questo linguaggio otteniamo delle buone fonetizzazioni per l'inglese.
\end{itemize}

\subsubsection{creazione di frasi}
L'ultimo passaggio è l'assemblaggio della frase, per questo problema ci possono essere due approcci risolutivi:

\paragraph{Flusso}
Con l'approccio di flusso consideriamo la frase come un'unica parola, ne consegue che non ci sono pause tra le parole ma solo alla fine della frase.

Per unire i singoli fonemi semplicemente sovrapponiamo le funzioni di due fonemi consecutivi e utilizziamo la funzione di transizione in sostituzione alla funzione di crescita e decrescita dei singoli fonemi.

\paragraph{Separazione}
Molto simile al caso precedente con l'unica differenza che consideriamo le parole come frasi e successivamente facciamo la transizione tra le parole.

\section {Graphik}
Questo engine si occupa della gestione grafica del programma. Questo modulo si interfaccia al processore grafico tramite OpenGL e consiste di tre parti principali:


\begin{itemize}
	\item Resource Manager  Questa parte ha lo scopo di gestire tutte le risorse grafiche che il programma utilizzerà.
	\item Input Manager Grazie a questo modulo è possibile accedere a svariati metodi di input tra cui Mouse e Tastiera.
	\item Renderers Questo modulo serve a comunicare alla GPU cosa disegnare.
\end{itemize}

\end{document}
