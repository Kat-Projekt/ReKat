\documentclass{article}
% librerie per lingua e fonts
\usepackage[italian]{babel}
\usepackage[utf8]{inputenc}
\usepackage[T1]{fontenc}

%sitture matematica
\usepackage{amsmath}
\usepackage{amsfonts}
\usepackage{amssymb}
\usepackage{amsthm}
\usepackage{yhmath}

% informazioni sul compito
\title{ReKat Ita}
\author{Eldron 64}
\date{November 2023}

\begin{document}
	
\maketitle
Con questo documento cercherò di mettere in ordine tutto quello che il Kat Projekt ora ReKat ha rappresentato per me.

\section{Introduzione}
Nell'implementazione corrente il Kat è un inisieme di Api che revono alla realizzazione di un progetto che speghero nel corso di qeusto rapporto.

Per il momento il Kat ha i seguenti componeneti:

\begin{itemize}
    \item Online: Node Network
    \item Synth: Sound System 
    \item Graphik: Graphics Api 
    \item LeT: Adactive learing System
\end{itemize}

Quando utilizziamo queste Api insieme otteniamo il Kat Projekt. Per come è progettata Graphik può controllare un robot: fornendo le specifiche ri rotazione a cui devono essere i vari motori per ottenere il risultato visibile a schermo.

\section {Online}
Questa Api è divisa in due implementazioni: 

\begin{itemize}
    \item Online Peer to Peer
    \item Online Server Side
\end{itemize}

Entrambe i Sistemi utilizziamo il metodo di criptazione RSA e si scambiano le chiavi pubbliche una volta connessi o al server o ad un nodo.

\subsection{Peer to Peer}
Questa implementazione si basa su un sistema di comunicazione Peer to Peer. Dove i vari nodi sono indipendeti tra loro e ogniuno possiede tutte le funzionalità degli altri nodi. 

L'utilizzo di questo modulo è molto semplice e lineare:

All'inizio del programma viene creato un Nodo locale a cui chiunque si può connettere. Succesivamente ci si può connettere ad un altro nodo nella rete.

I vari nodi possono eseguire dei programmi differenti ed offrire servizzi diversi. Nel caso un servizio non sia disponibile in un nodo a cui si è collegati si può eseguire una ricerca all'interno del Network per trovare il nodo con il programma richiesto ed accderci tramite i nodi intermediari.

\subsection{Server}
In questa implementazione il fulcro del Network è il server che esegue il programma principale a cui i vari client si connettono.

Questa implementazione rende più funzionale l'aggiurnamento della rete, visto che esiste un unico provider di connesione. Questo approccio però blocca la libertà nel codice bloccando alcuni servizzi al Server. 

Inoltre non è consentito il collegamento diretto tra i client, ogni comunicazione tra di loro passa per il Server. Questa implementazione è necessaria al fine di permettere la comunicazione tra client anche se il client ricevente non è connesso, infatti la comunicazione viene salvata nel Server per poi essere inoltrata al client.

\subsection{sicurezza RSA}
Ogni comunicazione uscente da un client, nodo o server viene criptata tramite il sistema RSA, viene inoltre criptata ad ogni passaggio. Quando il messaggio deve passare per un intermediario viene criptato due volta una per il destinatario e una per l'intermediario. In modo tale da fornire all'intermediario le informazioni necessarie a mandare il messagio.

Il paccheto che deve andare da A a B passando per C. verrà sottoposto ai seguenti passaggi:

\begin{itemize}
    \item A -> Pacchetto non criptato
    \item A -> Pacchetto criptato per B
    \item A -> Aggiunta al Pacchetto criptato le informazioni di spedizione per B
    \item A -> Nuovo Pacchetto criptato per C
    \item A-C -> A manda il pacchetto a C
    \item C -> Decripatazione del paccheto ricevuto
    \item C -> Lettura dal pacchetto l'indirizzo di B
    \item C -> Rimizione delle informazioni di B
    \item C-B -> C manda il pacchetto a B 
    \item B -> Decripatazione del pacchetto.
    \item B -> Pacchetto non criptato
\end{itemize}

Il sistema RSA si basa sull'avere una chiave pubblica a un privata, le chiavi sono formate da una parte comune N e una parte propria E (per la pubblica) e D (per la privata).
Questi valori sono utilizzati per eseguire operazioni matematiche ai dati che devono essere compresi tra 1 e N, se non rispettano questa condizione verrano corroti, per prevenire questa evenienza i dati vengono ridiemenzionati in sotto pacchetti di dimensione inferiore a N criptati e poi riuniti, questo processo funzione in entrambe le direzioni: criptaione <=> decripatazione.

\section {Synth}
Con questa libreria volgio fornire un modello di generazione sonora basato sul linquaggio umano e sui suoni che un essere umano può fare. Ci sono due appprocci a questo problema. 

Il primo consiste nella registrazione diretta di qualcuno che emette i suoni e alla loro sucessiva lavorazione e modululazione. Con questo aproccio possiamo estrarre le componenti basilari della dizione da una serie di campioni per poi adattari all'utilizzatore.

Il secondo metodo consiste nella simulazione di un aparato vocale, questo approccio è quello più complesso in termini di realizzazione ma è espandibile ad altri ambienti sonori.

\subsection{concetti di base}
Per iniziare dobbiamo definire un suono:

Un suono è una variazione di pressione che viene percepita da un apparato predisposto a misurare questa variazione.
In pratica un suono esiste quando qualcuno lo ascolta.

Un suono ha 4 componenti:
\begin{itemize}
    \item Il Tono
    \item La froma
    \item La durata
    \item La frequenza
\end{itemize}

Per semplificazione consideremo il suono come quantizzato. Il sample sonoro avrà una dimensione finita e può essere rappresentato come una serie di valori che rappresentatano la pressione dell'aria in quel preciso momento.

\subsubsection{il tono}
Il tono è il componente fondamentale di un suono. La più piccola parte periodica da cui è composto un suono.

Quando parliamo di vocalizzazione nell'essere umano possiamo dire che ogni fonema pronunciabile possiede un suo tono caratteristico che viene ripetuto per la durata della sua riproduzione e che ha come frequenza l'altezza del suono stesso.

Date queste precisazioni diamo al tono l'onere di dare ad ogni fonema il suo nome. Il fonema è influenzato dalla forma della bocca, dalla posizione della lingua, dalla nasalità del suono etc.

In termini matematici possiamo indicare $T(t)$ come la pressione dell'aria in quel momento, dobbiamo inoltre dire che $T(t)$ è periodica e per convenzione ha come intervallo di ripetizione $2\pi$. Può essere inoltre scomposta in una serie di $sin(t)$ e $cos(t)$ secondo la trasformata di Fuorier. $T(t)$ è una funzione normalizzara ovvero: $-1 \le t \le 1$.

\subsubsection{La froma}
La forma di un fonema è il modo in cui la sua intensità varia nel tempo ovvero come parte il suono, come viene mantenuto, e come decresce. 

Questo caratteristiche definiscono la pronuncia di un suono e un suo accento caratteristico.

Un suono puù crescere molto velocemente, può avre un effetto vibrato durante il mantenimento e può scendere molto lentamente.

La funzione della forma è composta secondo la seguente legge:

$$F(t) : 
\begin{cases}
    C(t)\hspace{0.2cm}per\hspace{0.2cm}t \le 1 \\
    M(t)\hspace{0.2cm}per\hspace{0.2cm}1 < t \le 2\\
    D(t)\hspace{0.2cm}per\hspace{0.2cm}2 < t 
\end{cases}
$$

Dove C,M,D sono le funzioni che indicano come $F(t)$ cresce nel tempo. Per convenzione e comodità di calcolo impongo $0 \le t \le 3$ per valori di X esterni al Dominio $F(t)$ restituisce un valore nullo. il valore di $t = 0$ indica l'inizio del suono mentre $t = 3$ ne indica la fine. $F(t)$ è una funzione normalizzara ovvero: $-1 \le t \le 1$.

\subsubsection{La durata}
Con la durata in tendiamo tre momenti: il tempo si crescita, il tempo di mantenimento e il tempo di decrescita.

Se con la forma descriviamo il modo in cui cresce e si modifica con la durata diamo informazioni su quanto tempo impiega per fare questa variazione.

La fuzione Durata: $H(t)$ indica la velocita con cui cresce la funzione. come $F(t)$ è definita a pezzi:
$$H(t) : 
\begin{cases}
    \frac{t}{t_1} \hspace{0.2cm}per\hspace{0.2cm} t \le t_1 \\
    \frac{t}{t_2} \hspace{0.2cm}per\hspace{0.2cm} t_1 < t \le t_2\\
    \frac{t}{t_3} \hspace{0.2cm}per\hspace{0.2cm} t_2 < t 
\end{cases}
$$
dove $t_1$, $t_2$ e $t_3$ sono i tempi caratteristici delle durate.

detto questo possiamo riscrivere $F(t)$

$$F_H(t) : 
\begin{cases}
    C(\frac{t}{t_1}) \hspace{0.2cm}per\hspace{0.2cm} t \le t_1 \\
    M(\frac{t}{t_2}) \hspace{0.2cm}per\hspace{0.2cm} t_1 < t \le t_2\\
    D(\frac{t}{t_3}) \hspace{0.2cm}per\hspace{0.2cm} t_2 < t 
\end{cases}
$$

con $0 \le t \le t_3$ e $t_1 < t_2 < t_3$

\subsubsection{La frequenza}
La frequenza descrive l'altezza di un suono ovvero quanto è acuto o gravi suddetto suono.

Per completezza devo tenere in conto ch' l'uomo può variare la frequenza di un fonema mentre lo esegue quindi:

Definiamo $f(t)$ la frequenza di un suono nell'intervallo $t$ questa semplice funzione crea un enormità di problemi relativi alla sincronizazione dei toni con frequenza diversa.
In pratica quello che fa questa funzione è modificare il periodo della funzione $T(t)$. $f(t)$ è definita strettamente positiva ovvero $f(t) > 0$.

Per semplificare impongo $f(t) = K$, ma verra sistemato in un iterazione sucessiva.

Ora dopo aver definito tutte le funzioni posso scrivere la funzione del Fonema come:

$$\Gamma(t) = F_H(t) T(2\pi f(t) t)$$

con $0 \le t \le t_3$.

\subsection{pronuncia}
Per fare in modo che una macchina pronunci correttamente una parola dobbiamo capire come noi pronunciamo le parole.

Noi come esseri umani possiamo emettere suoni singoli sotto la forma di fonemi o unire più fonemi ottenendo delle parole e con l'accostamento di più parole otteniamo una frase.

In pratica per pronunciare una parola bisogna pronunciare i vari fonemi che la compongono unendoli un qualche modo. Ma prima bisogna dividere le parole in fonemi.

\subsubsection{divisione fonetica}
Questo è un tema molto complesso. Infatti ogni lingua ha un suo sistema di dizione interno. Possiamo trovare varie categorie linguistiche che differiscono nel tipo di linguaggio fonetico che utilizzano.

Per quanto riguarda la divisione in fonemi bisogna implementatare un sistema specifico per ogni lingua. Di seguito riporterò alcuni esempi:

\begin{itemize}
    \item Italiano \\
        L'italiano ha una fonetizzazione abbastanza semplice, l'unica difficoltà risiede nelle lettere $g$ e $c$ che assumono suoni diversi in base al contesto: 
        \begin{itemize}
            \item $c$ e $g$ dolci: si hanno quando la lettera e seguita da $i$ o $e$; esempi: $nici$, $bici$, $ciao$, etc. 
            \item $c$ e $g$ dure: si hanno quando la lettera non è seguita direttamente da $i$ o $e$; esempi: $gola$, $lingua$, $ghiro$, etc.
        \end{itemize}
    \item Giapponese \\
        Il giapponese possiede una serie di ideogrammi che rappresentano tutte le possibili sillabe e con queste costruisono tutte le parole, questi caratteri vengono fonetizzati sempre allo stesso modo, ne consegue che basta constuire queste singole sillabe e poi unirle.
    \item Inglese \\
        L'inglese a differenza delle precedenti lingue presenta una fonetizzazione molto particolare, per fonetizzare l'inglese esite l' IPA (international Phonetic Alpabet), basandoci su questo linguaggio otteniamo delle buone fonetizzazioni dell'inglese.
\end{itemize}

\subsubsection{creazione di frasi}
L'ultimo passagio è l'assemblaggio della frase, per questo problema ci posso essere due approcci risolutivi: 

\paragraph{Flusso}
Con l'approccio di flusso consideriamo la frase come un unica parola, ne consegue che non ci sono pause tra le parole ma solo alla fine della frase.

Per unire i singoli fonemi semplicemente sovrapponiamo le funzioni di due fonemi consectivi e utilizziamo la funzione di transizione in sostituzione alla funzione di crescit e decrescita dei singoli fonemi.

\paragraph{Separazione}
Molto simile al caso precedente con l'unica differenza che consideriamo le parole come frasi e succesivamente facciamo la transizione tra le parole, ma con un intensità meno marcata.

\section {Graphik}
Questa sezione è un raggruppamento di varie librerie di grafica 3D. Lo scopo di questa API è quello di semplificare di gran lunga l'ultilizzo di queste librerie. E composta da vari moduli:

\begin{itemize}
    \item Resource Manager \\
        Questo modulo ha lo scopo di gestire tutte le risorse grafiche che il programma utilizzerà.
    \item Text Renderer \\
        Questo modulo serve per il rendereing di Testo, sia come pannelli assestanti ma anche sotto forma di mesh più complesse.
    \item Animator \\
        Questo modulo implementata l'animazione dei modelli può anche essere utilizzato per animare oggetti bidimensionali.
    \item Input Manager \\
        Grazie a questo modulo è possibile accedere a svariati input tra cui:
        \begin{itemize}
            \item Audio: sotto forma di microfono
            \item Meccanici: sotto forma di Maouse e tastiera
            \item Video: tramite l'utilizzo di una fotocamera
            \item Seriali: input provenienti da oggetti come arduino o rasberry pi
        \end{itemize}
    \item Phical Output \\
        Grazie a qeusto modulo si può comuincare direttamente con un arduino o rasberry pi appositamente confingurati. Questo serve a i risultati del programma nella realtà fisica.
\end{itemize}

\section {LeT}
E' un sistema di intelligenza Artificiale. Basato sulla persistenza delle memorie, riconosciemento di pattern e costante stimolo.

\subsection{Modello}
Il modello è diviso in quattro parti che si ripetono costantemente sempre nello stesso ordine.

\begin{itemize}
    \item Sensazioni
    \item Elaborazione
    \item Pensiero
    \item Impressioni
\end{itemize}

Questi passaggi riassumono quello che avviene nella nostra mente ogni secondo costante.

\paragraph{Sensazioni}
Le Sensazioni sono il metodo tramite il quale il modello riceve informazioni con l'esterno. Queste sensazioni sono: 
\begin{itemize}
    \item vista
    \item udito
    \item propriocezione, la percezione di se stessi ovvero come sono disposti i propri muscoli e come è orientato nello spazio.
    \item tatto 
    \item gusto e olfatto, raggiuppati in un unico senso considerando la loro natura affine
\end{itemize}

\paragraph{Elaborazione}
In questa seconda fase avviene l'elaborazione dei dati assimilati con la fase precedente secondo il seguente schema:
\begin{itemize}
    \item Tokenizzazione degli input; ovvero gli input vengono semplificati nei loro elementi fondamentali
    \item Confronto con il set precedente di input per determinare il movimento o una nuova interazione tra i Token
    \item Salvataggio dei Token; in questa parte vengono salvati i vari token e le relazioni che in quel momento sono presenti tra di essi
    \item Creazione di basici collegamenti tra Token: Associare le varie sensazioni ad un unico oggetto
\end{itemize}

\paragraph{Pensiero}
Questa è la sezione più importante è quella parte dove vengono elaborate le informazioni fornite dai processi di elaborazione, e vengono messe in relazione con ciò che il modello già conosce e permette di creare un nuovo modo di agire basato sulle proprie conoscienze.
I principali processi del pensiero sono:
\begin{itemize}
    \item relazioinare il contesto ad uno già noto
    \item cercare nel contesto le informazioni che lo differenziano da quello gia noto
    \item creazione di nuove memeorie dove vengono memorizati i concetti conosciuti
    \item cercare nelle memorie le informazioni richieste dal contesto
    \item cercare dei collegamenti tra i propri ricordi e quello che sta succedendo o altri ricordi correlati al presente.
    \item trovare il miglior modo per agire nella situazione presente
\end{itemize}

\paragraph{Impressioni}
Questo è l'ultimo passo e consiste nella possibilità del modello di esprimersi tramite diverse azioni tra cui l'emissione di suoni e il movimento.

\section {Kat Projekt}
Grazie a tutte le librerie definite sopra possiamo costruire un essere umano, in grado si vivere una vita come la nostra. Grazie a questa ricerca potremmo svelare il meccanismo nascosto dell nostra psiche e come noi influenziamo il mondo che ci circonda e come esso ci influenza.

Il fine ultimo di questa ricerca è costruire un lavoratore, un amico, una persona cara. 

Quello che voglio che rimanga da tutto questo è la consapevolezza di aver migliorato la vita di qualcuno, di avegli fornito qualcosa su cui contare in un modo pieno di incertezze come il nostro. 

Grazie per L'attenzione 

Elia

\end{document}
